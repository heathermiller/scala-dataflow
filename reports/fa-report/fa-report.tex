\documentclass[runningheads,a4paper,fleqn]{llncs}

\usepackage{url}
\usepackage{amsmath}
\usepackage{amssymb}
\usepackage{alltt}
\usepackage{lineno}
\usepackage{graphicx}
\usepackage[normalem]{ulem}

\usepackage{tikz}
\usetikzlibrary{arrows,petri,topaths}
\usepackage{tkz-berge}

\begin{document}

\title{Flow-Arrays: Barrier-Free Par-Arrays}
\author{Tobias Schlatter\inst{1} \and Aleksandar Prokopec\inst{2} \and
  Heather Miller\inst{2} \and Philipp Haller\inst{2} \and Martin
  Odersky\inst{2}}

\authorrunning{Tobias Schlatter}

\institute{Student, EPFL \and Advisors, EPFL}

\maketitle

\begin{abstract}
  In \cite{FP12} we proposed an unordered, barrier- and lock-free,
  parallel datastructure, the FlowPools. The following attempts to
  implement a bigger application using FlowPools, showed that the lack
  of ordering is very limiting. The proposed FlowArrays do not have
  this limitation: Having a very similar structure than ParArrays
  \cite{collect11}, FlowArrays eliminate the need to block inbetween
  monadic operations on ParArrays while exposing a very similar
  interface to the programmer.
  %% TODO write about performance
\end{abstract}

\section{Introduction}
The goal of this section is to briefly remind the reader of the
semantics and programming interface of a FlowPool and the basic ideas
behind the implementation of SLFPs to allow for better understanding
of the implementation of MLFPs.

\section{Implementation}

\section{Evaluation}

\section{Conclusion}

\bibliographystyle{abbrv}
\bibliography{bib}

\end{document}

%%% Local Variables: 
%%% mode: latex
%%% TeX-master: t
%%% End: 

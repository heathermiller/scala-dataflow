\documentclass{beamer}

\usepackage[utf8]{inputenc}

\usepackage{alltt}
\usepackage{xcolor}
\usepackage[overlay,absolute]{textpos}
\usepackage[normalem]{ulem}

\usepackage{tikz}
\usetikzlibrary{arrows,petri,topaths}
\usepackage{tkz-berge}

%% Colors for source highlighting
\definecolor{scKW}  {HTML}{AA37F2}
\definecolor{scFct} {HTML}{1010FF}
\definecolor{scType}{HTML}{228B22}
\definecolor{scVar} {HTML}{A45936}
\definecolor{scComm}{HTML}{B32525}

%% Define _s_cala commands
\newcommand{\sK}[1]{{\color{scKW} #1}}
\newcommand{\sF}[1]{{\color{scFct} #1}}
\newcommand{\sV}[1]{{\color{scVar} #1}}
\newcommand{\sT}[1]{{\color{scType} #1}}
\newcommand{\sC}[1]{{\color{scComm} #1}}
\newcommand{\sH}[1]{{\color{white} #1}}
\newcommand{\sN}[1]{{\color{black} #1}}
\newcommand{\sS}{\vspace{0.8mm}}

%% Define other commands
\newcommand<>{\strike}[1]{\alt#2{\sout{#1}}{#1}}

\setbeamercovered{transparent}
\setbeamertemplate{navigation symbols}

\setbeamertemplate{footline}{\makebox[0.98\paperwidth][r]{\large
    \raisebox{1.2ex}{\insertframenumber}}}

\graphicspath{{figs/}}

\title{FlowArrays}
\subtitle{Barrier-Free ParArrays}

\author{Tobias Schlatter}
\date{January 17, 2012}
\institute{Advisors: Aleksandar Prokopec, Heather Miller, Philipp Haller and Martin Odersky}

\begin{document}

\begin{frame}
  \titlepage
\end{frame}

\section{Introduction}

\begin{frame}
  \frametitle{Outline}
  
  \begin{block}{What is a FlowArray}\end{block}
  \begin{block}{Task Scheduling}\end{block}
  \begin{block}{Benchmarks}\end{block}

\end{frame}

\section{What is a FlowArray}

\begin{frame}
  \frametitle{What is a FlowArray}
  \framesubtitle{Big Picture}

  \begin{columns}[t]
    \column{.5\textwidth}

    FlowArray: Barrier-Free, single-assignment ParArray

    \vspace{\baselineskip}

    \begin{block}{FlowArray Properties}
      \begin{itemize}
      \item<1> IndexedSeq semantics
      \item<2,3> Asynchronous
      \item<4> Deterministic
      \item<5> Lock-Free
      \end{itemize}
    \end{block}

    \column{.5\textwidth}

    \begin{overprint}
      \onslide<1>
      \begin{block}{IndexedSeq semantics}
        \begin{itemize}
        \item Immutable
        \item Ordered
        \item Fixed size
        \item Monadic operations\\
          \texttt{map}, \texttt{fold}, \texttt{flatMap}, etc.
        \end{itemize}
      \end{block}

      \onslide<2>
      \includegraphics[page=1]{async}

      \onslide<3>
      \includegraphics[page=2]{async}

      \onslide<4>
      \begin{block}{Determinism}
        Every execution of a given program with given input eventually
        \begin{itemize}
        \item Always reaches same state
        \item[] \qquad or
        \item Always fails
        \end{itemize}
      \end{block}

      \onslide<5>
      \begin{block}{Locking}
        \begin{alltt}
          \sK{synchronized} \{\\
          \sH{xx}i = i + 1\\
          \}
        \end{alltt}
      \end{block}

      \begin{block}{Lock-Free}
        \begin{alltt}
          \sK{do} \{\\
          \sH{xx}ov = READ(i)\\
          \sH{xx}nv = ov + 1\\
          \} \sK{while} (!CAS(i, ov, nv))
        \end{alltt}
      \end{block}

    \end{overprint}

  \end{columns}

\end{frame}

\begin{frame}
  \frametitle{What is a FlowArray}
  \framesubtitle{Programming Model -- ParArray}

  \begin{alltt} \small
    \sK{val} \sV{pa1} = ParArray.tabulate(100)(x => x * x)\\
    \uncover<1,11->{\sK{val} \sV{pa2} = pa1.map(\_ * 2)}\\
    \uncover<1,14->{\sK{val} \sV{el}\phantom{xx}= pa2(35)}
  \end{alltt}

  \vspace{\stretch{1}}

  \begin{overprint}
    \onslide<2>\includegraphics[page=1]{pa-map}
    \onslide<3>\includegraphics[page=2]{pa-map}
	\onslide<4>\includegraphics[page=3]{pa-map}
	\onslide<5>\includegraphics[page=4]{pa-map}
	\onslide<6>\includegraphics[page=5]{pa-map}
	\onslide<7>\includegraphics[page=6]{pa-map}
	\onslide<8>\includegraphics[page=7]{pa-map}
	\onslide<9>\includegraphics[page=8]{pa-map}
	\onslide<10>\includegraphics[page=9]{pa-map}
	\onslide<11>\includegraphics[page=10]{pa-map}
	\onslide<12>\includegraphics[page=11]{pa-map}
	\onslide<13->\includegraphics[page=12]{pa-map}
  \end{overprint}

\end{frame}

\begin{frame}
  \frametitle{What is a FlowArray}
  \framesubtitle{Programming Model -- FlowArray}

  \begin{alltt} \small
    \sK{val} \sV{fa1} = FlowArray.tabulate(100)(x => x * x)\\
    \uncover<1,3->{\sK{val} \sV{fa2} = fa1.map(\_ * 2)}\\
    \uncover<1,5->{\sK{val} \sV{el}\phantom{xx}= fa2.blocking(35)}
  \end{alltt}

  \vspace{\stretch{1}}

  \begin{overprint}
    \onslide<2,3>\includegraphics[page=1]{fa-map}
    \onslide<4>\includegraphics[page=2]{fa-map}
    \onslide<5>\includegraphics[page=3]{fa-map}
    \onslide<6>\includegraphics[page=4]{fa-map}
    \onslide<7>\includegraphics[page=5]{fa-map}
    \onslide<8>\includegraphics[page=6]{fa-map}
    \onslide<9>\includegraphics[page=7]{fa-map}
    \onslide<10>\includegraphics[page=8]{fa-map}
    \onslide<11>\includegraphics[page=9]{fa-map}
    \onslide<12>\includegraphics[page=10]{fa-map}
    \onslide<13>\includegraphics[page=11]{fa-map}
    \onslide<14>\includegraphics[page=12]{fa-map}
  \end{overprint}

\end{frame}

\section{Task Scheduling}

\begin{frame}
  \frametitle{Task Scheduling}

\end{frame}

\section{Benchmarks}

\begin{frame}
  \frametitle{Benchmarks}
  \framesubtitle{CPU-Scaling}

  %% Hack to use whole frame
  \begin{columns}[c]
    \begin{column}{\paperwidth}
      \centering
      %\includegraphics[trim = 35mm 0mm 36mm 0mm, clip, width=0.95\paperwidth]
      %{../../benchmarks/pres_graphs/cpu-scaling-insert}
      \par
    \end{column}
  \end{columns}

\end{frame}

\section{Conclusion}

\begin{frame}
  \frametitle{Conclusion}

  {\Large FlowPools are \ldots}

  \pause
  \begin{block}{Basic Traits}
    \begin{itemize}
    \item Flow-Based Collection
    \item Asynchronous
    \item Deterministic
    \end{itemize}
  \end{block}
      
  \pause
  \begin{block}{Performance \& Scalability}
    \begin{itemize}
    \item Speed comparable to Java standard Queues
    \item Scalable
    \item Unneeded Elements GC'd
    \end{itemize}
  \end{block}

  \pause
  \begin{block}{¿Questions?}
  \end{block}

\end{frame}

\section{Appendix}

\begin{frame}
  \frametitle{Details about Benchmarks}

  \begin{columns}[t]
    \begin{column}{.5\textwidth}
      \begin{block}{Insert / Reduce}
        \begin{itemize}
        \item $5\cdot10^6$ elements
        \item 20 measurements
        \end{itemize}
      \end{block}

      \begin{block}{Communication / GC}
        \begin{itemize}
        \item Parallelization level: 1
        \item Measurements
          \begin{itemize}
          \item UltraSPARC T2: 2
          \item 4-core i7: 10
          \item 32-core Xeon: 3
          \end{itemize}
        \end{itemize}
      \end{block}

    \end{column}
    \begin{column}{.5\textwidth}
      \begin{block}{Java Command}
        \small\texttt{-Xmx2048m -Xms2048m -XX:+UseCondCardMark
          -verbose:gc -XX:+PrintGCDetails -server}.
      \end{block}
      \begin{block}{Java Version}
        \begin{itemize}
        \item Intel \small\texttt{
            1.7.0\_04-ea-b15\\
            HotSpot 64-Bit Server VM (build 23.0-b16, mixed mode)
          }
        \item SPARC \small\texttt{
            1.7.0\_03-b04\\
            HotSpot Server VM (build 22.1-b02, mixed mode)
          }
        \end{itemize}
      \end{block}
    \end{column}
  \end{columns}
    
\end{frame}

\begin{frame}
  \frametitle{Details about Benchmarks}

  \begin{block}{Architectures}
    \begin{itemize}
    \item octa-core 1.2GHz UltraSPARC T2 w/ 64 HW threads
    \item quad-core 3.4 GHz i7-2600 (w/ HT)
    \item 4x octa-core 2.27 GHz Intel Xeon x7560 (w/ HT)
    \end{itemize}
  \end{block}

\end{frame}

\end{document}

%%% Local Variables: 
%%% mode: latex
%%% TeX-master: t
%%% End: 
